\chapter{User defined functions}\label{ExprTk}

{\em Much of this chapter is exerpted from \htmladdnormallink{exprtk's
read.txt
file}{https://github.com/ArashPartow/exprtk/blob/master/readme.txt}}

\section{Introduction}
The \htmladdnormallink{C++ Mathematical Expression  Toolkit Library (ExprTk)}{https://www.partow.net/programming/exprtk/index.html} is  a simple
to  use,   easy  to   integrate  and   extremely  efficient   run-time
mathematical  expression parsing  and evaluation  engine. The  parsing
engine  supports numerous  forms  of  functional and  logic processing
semantics and is easily extensible.

With Minsky's user defined functions, expressions can refer to Minsky
variables accessible from the current scope (ie local Minsky variables
will hide global variables), and also parameters declared as part of
the function name. One can also call other user defined functions,
which is the only way a user defined function with more than 2
parameters can be used. For 0-2 parameters, user defined functions can
be wired into a Minsky computation.

ExprTk identifiers (such as variable names and function names) consist
of alphanumeric characters plus '\_' and '.'. They must start with a
letter. Minsky is reserving the underscore and full stop to act as an
escape sequence, in order to refer to the full range of possible
Minsky variable identifiers, including all unicode characters. This
section will be updated once that feature is in place --- for now,
please avoid using those characters in identifiers.

\section{Capabilities}
The  ExprTk expression  evaluator supports  the following  fundamental
arithmetic operations, functions and processes:

\begin{description}
 \item[Types:]           Scalar, Vector, String

 \item[Basic operators:] \verb'+, -, *, /, %, ^'

 \item[Assignment:]      \verb':=, +=, -=, *=, /=, %='

 \item[Equalities \& Inequalities:]    \verb'=, ==, <>, !=, <, <=, >, >='

 \item[Logic operators:] and, mand, mor, nand, nor, not, or, shl, shr,
                       xnor, xor, true, false

 \item[Functions:]       abs, avg, ceil, clamp, equal, erf, erfc,  exp,
                       expm1, floor, frac,  log, log10, log1p,  log2,
                       logn,  max,  min,  mul,  ncdf,  nequal,  root,
                       round, roundn, sgn, sqrt, sum, swap, trunc

 \item[Trigonometry:]    acos, acosh, asin, asinh, atan, atanh,  atan2,
                       cos,  cosh, cot,  csc, sec,  sin, sinc,  sinh,
                       tan, tanh, hypot, rad2deg, deg2grad,  deg2rad,
                       grad2deg

 \item[Control structures:]      if-then-else, ternary conditional, switch-case,
                       return-statement

 \item[Loop statements:] while, for, repeat-until, break, continue

 \item[String processing:]      in, like, ilike, concatenation

 \item[Optimisations:]   constant-folding, simple strength reduction and
                       dead code elimination

 \item[Calculus:]        numerical integration and differentiation

\end{description}

\section{Example expressions}
The following is  a short listing  of infix format  based mathematical
expressions that can be parsed and evaluated using the ExprTk library.

\begin{itemize}
  \item \verb'sqrt(1 - (3 / x^2))'
  \item \verb'clamp(-1, sin(2 * pi * x) + cos(y / 2 * pi), +1)'
  \item \verb'sin(2.34e-3 * x)'
  \item \verb'if(((x[2] + 2) == 3) and ((y + 5) <= 9),1 + w, 2 / z)'
  \item \verb'inrange(-2,m,+2) == if(({-2 <= m} and [m <= +2]),1,0)'
  \item \verb'({1/1}*[1/2]+(1/3))-{1/4}^[1/5]+(1/6)-({1/7}+[1/8]*(1/9))'
  \item \verb'a * exp(2.2 / 3.3 * t) + c'
  \item \verb'z := x + sin(2.567 * pi / y)'
  \item \verb'u := 2.123 * {pi * z} / (w := x + cos(y / pi))'
  \item \verb'2x + 3y + 4z + 5w == 2 * x + 3 * y + 4 * z + 5 * w'
  \item \verb'3(x + y) / 2.9 + 1.234e+12 == 3 * (x + y) / 2.9 + 1.234e+12'
  \item \verb'(x + y)3.3 + 1 / 4.5 == [x + y] * 3.3 + 1 / 4.5'
  \item \verb'(x + y[i])z + 1.1 / 2.7 == (x + y[i]) * z + 1.1 / 2.7'
  \item \verb'(sin(x / pi) cos(2y) + 1) == (sin(x / pi) * cos(2 * y) + 1)'
  \item \verb'75x^17 + 25.1x^5 - 35x^4 - 15.2x^3 + 40x^2 - 15.3x + 1'
  \item \verb'(avg(x,y) <= x + y ? x - y : x * y) + 2.345 * pi / x'
  \item \verb'while (x <= 100) { x -= 1; }'
  \item \verb"x <= 'abc123' and (y in 'AString') or ('1x2y3z' != z)"
  \item \verb"((x + 'abc') like '*123*') or ('a123b' ilike y)"
  \item \verb'sgn(+1.2^3.4z / -5.6y) <= {-7.8^9 / -10.11x }'
\end{itemize}

\section{Copyright notice}
Free  use  of  the  C++  Mathematical  Expression  Toolkit  Library is
permitted under the guidelines and in accordance with the most current
version of the \htmladdnormallink{MIT License}{http://www.opensource.org/licenses/MIT}


\section{Built-in operations \& functions}

\subsection{Arithmetic \& Assignment Operators}

\begin{tabular}{|c|p{0.8\textwidth}|}
\hline
OPERATOR & DEFINITION \\
\hline
\verb'+' &  Addition between x and y.  (eg: \verb'x + y')\\
\verb'-'& Subtraction between x and y.  (eg: \verb'x - y')\\
\verb'*'&   Multiplication between x and y.  (eg: \verb'x * y')\\
\verb'/'& Division between x and y.  (eg: \verb'x / y')\\
\verb'%'& Modulus of x with respect to y.  (eg: \verb'x % y')\\
\verb'^'& $x^y$.  (eg: \verb'x ^ y')\\
\verb':='& Assign the value of x to y. Where y is either a variable
 or vector type.  (eg: \verb'y := x')\\
\verb'+='&Increment x by the value of the expression on the right 
hand side. Where x is either a variable or vector type. 
(eg: \verb'x += abs(y - z)')\\
\verb'-='&  Decrement x by the value of the expression on the right 
hand side. Where x is either a variable or vector type. 
(eg: \verb'x[i] -= abs(y + z)')\\
\verb'*='& Assign the multiplication of x by the value of the 
 expression on the righthand side to x. Where x is either
 a variable or vector type. (eg: \verb'x *= abs(y / z)')\\
\verb'/='& Assign the division of x by the value of the expression
on the right-hand side to x. Where x is either a  variable or vector
type.  (eg: \verb'x[i + j] /= abs(y * z)')\\
\verb'%='& Assign x modulo the value of the expression on the right
hand side to x. Where x is either a variable or vector type.  (eg:
\verb'x[2] %= y ^ 2')\\
\hline
\end{tabular}

\subsection{Equalities \& Inequalities}

\begin{tabular}{|c|p{0.8\textwidth}|}
\hline
OPERATOR & DEFINITION\\
\hline
\verb'==' or \verb'=' & True only if x is strictly equal to y. (eg: \verb'x == y')\\
\verb'<>' or \verb'!='& True only if x does not equal y. (eg: \verb'x <> y' or \verb'x != y')\\
\verb'<'& True only if x is less than y. (eg: \verb'x < y')\\
\verb'<='& True only if x is less than or equal to y. (eg: \verb'x <= y')\\
\verb'>'& True only if x is greater than y. (eg: \verb'x > y'\\
\verb'>='& True only if x greater than or equal to y. (eg: \verb'x >= y')\\
\hline
\end{tabular}

\subsection{Boolean Operations}

\begin{tabular}{|c|p{0.8\textwidth}|}
\hline
OPERATOR& DEFINITION\\
\hline
\verb'true'& True state or any value other than zero (typically 1).\\
\verb'false'& False state, value of exactly zero.\\
\verb'and'& Logical AND, True only if x and y are both true. (eg: \verb'x and y')\\
\verb'mand'& Multi-input logical AND, True only if all inputs are true. Left to right short-circuiting of expressions. (eg: \verb'mand(x > y, z < w, u or v, w and x)')\\
\verb'mor'& Multi-input logical OR, True if at least one of the inputs
are true. Left to right short-circuiting of expressions.  (eg:
\verb'mor(x > y, z < w, u or v, w and x)')\\
\verb'nand'& Logical NAND, True only if either x or y is false. (eg: \verb'x nand y')\\
\verb'nor'& Logical NOR, True only if the result of x or y is false (eg: \verb'x nor y')\\
\verb'not'& Logical NOT, Negate the logical sense of the input. (eg: \verb'not(x and y) == x nand y')\\
\verb'or'& Logical OR, True if either x or y is true. (eg: \verb'x or y') \\
\verb'xor'& Logical XOR, True only if the logical states of x and y differ.  (eg: \verb'x xor y')\\
\verb'xnor'& Logical XNOR, True iff the biconditional of x and y is
satisfied.  (eg: \verb'x xnor y')\\
\verb'&'& Similar to AND but with left to right expression short circuiting optimisation.  (eg: \verb'(x & y) == (y and x)')\\
\verb'|'& Similar to OR but with left to right expression short
circuiting optimisation.  (eg: \verb'(x | y) == (y or x)')\\
\hline
\end{tabular}

\subsection{General Purpose Functions}

\begin{tabular}{|c|p{0.8\textwidth}|}
\hline
FUNCTION & DEFINITION\\
\hline
\verb'abs'& Absolute value of x.  (eg: \verb'abs(x)')\\
\verb'avg'& Average of all the inputs. (eg: \verb'avg(x,y,z,w,u,v) == (x + y + z + w + u + v) / 6')\\
\verb'ceil'& Smallest integer that is greater than or equal to x.\\
\verb'clamp'& Clamp x in range between r0 and r1, where r0 < r1. (eg:
\verb'clamp(r0,x,r1)')\\
\verb'equal'& Equality test between x and y using normalised epsilon\\
\verb'erf'& Error function of x.  (eg: \verb'erf(x)')\\
\verb'erfc'& Complimentary error function of x.  (eg: \verb'erfc(x)')\\
\verb'exp'& $e^x$  (eg: \verb'exp(x)')\\
\verb'expm1'& $e^{x-1}$ where $x$ is very small. (eg: \verb'expm1(x)')\\ 
\verb'floor'& Largest integer that is less than or equal to x. (eg: \verb'floor(x)')\\
\verb'frac'& Fractional portion of x.  (eg: \verb'frac(x)')\\
\verb'hypot'& $\sqrt{x^2+y^2}$ (eg: \verb'hypot(x,y) = sqrt(x*x + y*y)')\\
\verb'iclamp'& Inverse-clamp x outside of the range r0 and r1. Where
r0 < r1. If x is within the range it will snap to the closest
bound. (eg: \verb'iclamp(r0,x,r1)'
\begin{math}
=\left\{\begin{array}{ccc}
r0 & \mathrm{if} & x\le r0\\
x & \mathrm{if} & r0\le x \le r1\\
r1 & \mathrm{if} & x\ge r1\\
\end{array}\right.
\end{math}
)\\
\verb'inrange'&  In-range returns 'true' when $x$ is within the range $[r_0,r_1]$. Where $r_0 < r_1$.  (eg: \verb'inrange(r0,x,r1)')\\
\verb'log'& Natural logarithm $\ln x$.  (eg: \verb'log(x)')\\
\verb'log10'& $\log_{10}x$.  (eg: \verb'log10(x)')\\
\verb'log1p'& $\ln (1+x)$, where $x$ is very small. (eg: \verb'log1p(x)')\\
\verb'log2'& $\log_2x$.  (eg: \verb'log2(x)')\\
\verb'logn'& $\log_nx$, where $n$ is a positive integer. (eg: \verb'logn(x,8)')\\
\verb'max'& Largest value of all the inputs. (eg: \verb'max(x,y,z,w,u,v)')\\
\verb'min'& Smallest value of all the inputs. (eg: \verb'min(x,y,z,w,u)')\\
\verb'mul'& Product of all the inputs. (eg: \verb'mul(x,y,z,w,u,v,t) == (x * y * z * w * u * v * t)') \\
\verb'ncdf'& Normal cumulative distribution function.  (eg: \verb'ncdf(x)')\\
\verb'nequal'& Not-equal test between $x$ and $y$ using normalised epsilon\\
\verb'pow'& $x^y$.  (eg: \verb'pow(x,y) == x ^ y')\\
\verb'root'&  $\sqrt[n]{x}$, where $n$ is a positive integer. (eg: \verb'root(x,3) == x^(1/3)')\\
\verb'round'& Round $x$ to the nearest integer.  (eg: \verb'round(x)')\\
\verb'roundn'& Round $x$ to $n$ decimal places  (eg: \verb'roundn(x,3)')
 where $n > 0$ is an integer. (eg: \verb'roundn(1.2345678,4) == 1.2346')\\\verb'sgn'& Sign of $x$, $-1$ where $x < 0$, +1 where $x > 0$, else zero.
 (eg: \verb'sgn(x)')\\
\verb'sqrt'& $\sqrt{x}$, where $x >= 0$.  (eg: \verb'sqrt(x)')\\ 
\verb'sum'& Sum of all the inputs. (eg: \verb'sum(x,y,z,w,u,v,t) == (x + y + z + w + u + v + t)')\\
\verb'swap'\\\verb'<=>'& Swap the values of the variables x and y and return the current value of y.  (eg: \verb'swap(x,y)' or \verb'x <=> y')\\
\verb'trunc'& Integer portion of x.  (eg: \verb'trunc(x)')\\
\hline
\end{tabular}

\subsection{Trigonometry Functions}

\begin{tabular}{|c|p{0.8\textwidth}|}
\hline
FUNCTION & DEFINITION\\
\hline
\verb'acos'& Arc cosine of x expressed in radians. Interval $[-1,+1]$
(eg: \verb'acos(x)')\\
\verb'acosh'& Inverse hyperbolic cosine of x expressed in radians. (eg: \verb'acosh(x)')\\
\verb'asin'& Arc sine of x expressed in radians. Interval $[-1,+1]$ (eg: \verb'asin(x)')\\
\verb'asinh'& Inverse hyperbolic sine of x expressed in radians. (eg:
\verb'asinh(x)')\\
\verb'atan'& Arc tangent of x expressed in radians. Interval $[-1,+1]$
(eg: \verb'atan(x)')\\
\verb'atan2'& Arc tangent of $(x / y)$ expressed in
radians. $[-\pi,+\pi]$ (eg: \verb'atan2(x,y)')\\
\verb'atanh'& Inverse hyperbolic tangent of $x$ expressed in radians. (eg: \verb'atanh(x)')\\
\verb'cos'& Cosine of $x$.  (eg: \verb'cos(x)')\\
\verb'cosh'& Hyperbolic cosine of $x$.  (eg: \verb'cosh(x)')\\
\verb'cot'& Cotangent of $x$.  (eg: \verb'cot(x)')\\
\verb'csc'& Cosecant of $x$.  (eg: \verb'csc(x)')\\ 
\verb'sec'& Secant of $x$.  (eg: \verb'sec(x)')\\
\verb'sin'& Sine of $x$.  (eg: \verb'sin(x)')\\
\verb'sinc'& Sine cardinal of $x$.  (eg: \verb'sinc(x)')\\
\verb'sinh'& Hyperbolic sine of $x$.  (eg: \verb'sinh(x)')\\
\verb'tan'& Tangent of $x$.  (eg: \verb'tan(x)')\\
\verb'tanh'& Hyperbolic tangent of $x$.  (eg: \verb'tanh(x)')\\ 
\verb'deg2rad'& Convert $x$ from degrees to radians.  (eg: \verb'deg2rad(x)')\\
\verb'deg2grad'& Convert $x$ from degrees to gradians.  (eg: \verb'deg2grad(x)')\\
\verb'rad2deg'& Convert $x$ from radians to degrees.  (eg: \verb'rad2deg(x)')\\
\verb'grad2deg'& Convert $x$ from gradians to degrees.  (eg: \verb'grad2deg(x)')\\
\hline
\end{tabular}

\subsection{String Processing}
\begin{tabular}{|c|p{0.8\textwidth}|}
\hline
FUNCTION& DEFINITION\\
\verb'=' , \verb'==', \verb'!=', \verb'<>', \verb'<=', \verb'>=', \verb'<' , \verb'>'& All common equality/inequality operators are applicable to strings and are applied in a case sensitive manner. In the following example x, y and z are of type string. (eg: \verb"not((x <= 'AbC') and ('1x2y3z' <> y)) or (z == x)")\\
\verb'in'& True only if $x$ is a substring of $y$. (eg: \verb'x in y' or \verb"'abc' in 'abcdefgh'")\\
\verb'like'& True only if the string x matches the pattern y.  Available wildcard characters are `*' and `?' denoting  zero or more and zero or one matches respectively.   (eg: \verb'x like y' or \verb"'abcdefgh' like 'a?d*h'")\\ 
\verb'ilike'& True only if the string x matches the pattern y in a case insensitive manner. Available wildcard characters are '*' and '?' denoting zero or more and zero or one  matches respectively. (eg: \verb'x ilike y' or \verb"'a1B2c3D4e5F6g7H' ilike 'a?d*h'")\\
\verb'[r0:r1]'& The closed interval $[r0,r1]$ of the specified string.
eg: Given a string x with a value of 'abcdefgh' then:
\begin{enumerate}
\item \verb"x[1:4] == 'bcde'"
\item \verb"x[ :5] == x[:10 / 2] == 'abcdef'"
\item \verb"x[2 + 1: ] == x[3:] =='defgh'"
\item \verb"x[ : ] == x[:] == 'abcdefgh'"
\item \verb"x[4/2:3+2] == x[2:5] == 'cdef'"
\end{enumerate}
 Note: Both r0 and r1 are assumed to be integers, where r0 <= r1. They
 may also be the result of an expression, in the event they have
 fractional components truncation will be performed. (eg: $1.67
                \rightarrow 1$)\\
  %begin{latexonly}
  \hline
\end{tabular}

\begin{tabular}{|c|p{0.8\textwidth}|}
\hline
FUNCTION& DEFINITION\\
  \hline
  %end{latexonly}
\verb':='& Assign the value of x to y. Where y is a mutable string  or
 string range and x is either a string or a string  range. eg:
 \begin{enumerate}
\item\verb'y := x'
\item\verb"y := 'abc'"
\item\verb'y := x[:i + j]'
\item\verb"y := '0123456789'[2:7]"
\item\verb"y := '0123456789'[2i + 1:7]"
\item\verb"y := (x := '0123456789'[2:7])"
\item\verb"y[i:j] := x"
\item\verb"y[i:j] := (x + 'abcdefg'[8 / 4:5])[m:n]"
\end{enumerate}

Note: For options 7 and 8 the shorter of the two ranges 
will denote the number characters that are to be copied.\\
\verb'+'&  Concatenation of x and y. Where x and y are strings or  
 string ranges. eg
 \begin{enumerate}
\item\verb"x + y"
\item\verb"x + 'abc'"
\item\verb"x + y[:i + j]"
\item\verb"x[i:j] + y[2:3] + '0123456789'[2:7]"
\item\verb"'abc' + x + y"
\item\verb"'abc' + '1234567'"
\item\verb"(x + 'a1B2c3D4' + y)[i:2j]"
\end{enumerate}\\
\verb'+='& Append to x the value of y. Where x is a mutable string 
and y is either a string or a string range. eg:
\begin{enumerate}
\item\verb"x += y"                  
\item\verb"x += 'abc'"              
\item\verb"x += y[:i + j] + 'abc'"
\item\verb"x += '0123456789'[2:7]"
\end{enumerate}
\\
\verb'<=>'&  Swap the values of x and y. Where x and y are mutable   
            strings.  (eg: \verb'x <=> y')\\
  %begin{latexonly}
  \hline
\end{tabular}

\begin{tabular}{|c|p{0.8\textwidth}|}
\hline
FUNCTION& DEFINITION\\
  \hline
  %end{latexonly}
\verb'[]'& The string size operator returns the size of the string 
 being actioned. eg:
\begin{enumerate}
\item\verb"'abc'[] == 3"
\item\verb"var max_str_length := max(s0[],s1[],s2[],s3[]"
\item\verb"('abc' + 'xyz')[] == 6"
\item\verb"(('abc' + 'xyz')[1:4])[] == 4"
\end{enumerate}\\
  \hline
\end{tabular}

\subsection{Control Structures}
\begin{tabular}{|c|p{0.8\textwidth}|}
  \hline
STRUCTURE & DEFINITION\\
\verb'if'& If x is true then return y else return z.eg:
\begin{enumerate}
\item\verb"if (x, y, z)" 
\item\verb"if ((x + 1) > 2y, z + 1, w / v)"
\item\verb"if (x > y) z;"
\item\verb"if (x <= 2*y) { z + w };"
\end{enumerate}\\
\verb'if-else'& The if-else/else-if statement. Subject to the condition 
branch the statement will return either the value of the consequent or
the alternative branch. eg:
\begin{enumerate}
\item\verb"if (x > y) z; else w;"
\item\verb"if (x > y) z; else if (w != u) v;"
\item\verb"if (x < y) { z; w + 1; } else u;"
\item
\begin{verbatim}
if ((x != y) and (z > w))
{   
  y := sin(x) / u;
  z := w + 1;     
}                  
else if (x > (z + 1))
{                    
  w := abs (x - y) + z;
  u := (x + 1) > 2y ? 2u : 3u;
}
\end{verbatim}
\end{enumerate}\\

\verb'switch'& The first true case condition that is encountered will 
determine the result of the switch. If none of the case
conditions hold true, the default action is assumed as 
the final return value. This is sometimes also known as
a multi-way branch mechanism.                          
eg:
\begin{verbatim}
switch                                                 
{                                                      
  case x > (y + z) : 2 * x / abs(y - z);               
  case x < 3       : sin(x + y);                       
  default          : 1 + x;                            
}            
\end{verbatim}
\\
\verb'while'& The structure will repeatedly evaluate the internal     
statement(s) 'while' the condition is true. The final  
statement in the final iteration will be used as the   
return value of the loop.                              
eg:                                                    
\begin{verbatim}
while ((x -= 1) > 0)                                   
{                                                      
  y := x + z;                                          
  w := u + y;                                          
}               
\end{verbatim}
  \\
  %begin{latexonly}
  \hline
\end{tabular}

\begin{tabular}{|c|p{0.8\textwidth}|}
\hline
FUNCTION& DEFINITION\\
  \hline
  %end{latexonly}
\verb'repeat/until'& The structure will repeatedly evaluate the internal
statement(s) 'until' the condition is true. The final
statement in the final iteration will be used as the 
return value of the loop.                            
eg:
\begin{verbatim}
repeat                                               
  y := x + z;                                        
  w := u + y;                                        
until ((x += 1) > 100)                               
\end{verbatim}
\\
\verb'for'& The structure will repeatedly evaluate the internal    
statement(s) while the condition is true. On each loop
iteration, an 'incrementing' expression is evaluated. 
The conditional is mandatory whereas the initialiser  
and incrementing expressions are optional.            
eg:                                                   
\begin{verbatim}
for (var x := 0; (x < n) and (x != y); x += 1)        
{                                                     
  y := y + x / 2 - z;                                 
  w := u + y;                                         
}            
\end{verbatim}
\\
\verb'break/break[]'& Break terminates the execution of the nearest enclosed 
loop, allowing for the execution to continue on external
to the loop. The default break statement will set the   
return value of the loop to NaN, where as the return    
based form will set the value to that of the break      
expression.                                             
eg:
\begin{verbatim}
while ((i += 1) < 10)                                   
{                                                       
  if (i < 5)                                            
    j -= i + 2;                                         
  else if (i % 2 == 0)                                  
    break;                                              
  else                                                  
    break[2i + 3];                                      
}            
\end{verbatim}
\\
\verb'continue'& Continue results in the remaining portion of the nearest
enclosing loop body to be skipped.
eg:
\begin{verbatim}
for (var i := 0; i < 10; i += 1)  
{                                 
  if (i < 5)                      
    continue;                     
  j -= i + 2;                     
}            
\end{verbatim}
  \\
  %begin{latexonly}
  \hline
\end{tabular}

\begin{tabular}{|c|p{0.8\textwidth}|}
\hline
FUNCTION& DEFINITION\\
  \hline
  %end{latexonly}
\verb'return'& Return immediately from within the current expression.
With the option of passing back a variable number of
values (scalar, vector or string). eg:
\begin{enumerate}
\item\verb"return [1]; "                                     
\item\verb"return [x, 'abx'];"                              
\item\verb"return [x, x + y,'abx'];"                         
\item\verb"return [];"                                       
\item
\begin{verbatim}
if (x < y)                                       
    return [x, x - y, 'result-set1', 123.456];      
   else                                             
    return [y, x + y, 'result-set2'];               
\end{verbatim}
\end{enumerate}
\\
\verb'?:'& Ternary conditional statement, similar to that of the
above denoted if-statement.                     
eg:
\begin{enumerate}
\item\verb"x ? y : z"                                    
\item\verb"x + 1 > 2y ? z + 1 : (w / v)"                 
\item\verb"min(x,y) > z ? (x < y + 1) ? x : y : (w * v)"
\end{enumerate}
\\
\verb'~'& Evaluate each sub-expression, then return as the result 
the value of the last sub-expression. This is sometimes
known as multiple sequence point evaluation.           
eg:                                                    
\begin{verbatim}
~(i := x + 1, j := y / z, k := sin(w/u)) == (sin(w/u)))
~{i := x + 1; j := y / z; k := sin(w/u)} == (sin(w/u)))
\end{verbatim}
\\
\verb'[*]'& Evaluate any consequent for which its case statement is 
true. The return value will be either zero or the result
of the last consequent to have been evaluated.          
eg:
\begin{verbatim}
[*]                                                     
{                                                       
  case (x + 1) > (y - 2)    : x := z / 2 + sin(y / pi); 
  case (x + 2) < abs(y + 3) : w / 4 + min(5y,9);        
  case (x + 3) == (y * 4)   : y := abs(z / 6) + 7y;     
}              
\end{verbatim}
\\
\verb'[]'& The vector size operator returns the size of the vector 
being actioned.                         
eg:
\begin{enumerate}
\item\verb"v[]"                                 
\item\verb"max_size := max(v0[],v1[],v2[],v3[])"
\end{enumerate}
\\
\hline
\end{tabular}

Note: In  the  tables  above, the  symbols x, y, z, w, u  and v  where
appropriate may represent any of one the following:

\begin{enumerate}
  \item Literal numeric/string value
   \item A variable
   \item A vector element
   \item A vector
   \item A string
   \item An expression comprised of [1], [2] or [3] (eg: \verb'2 + x /vec[3])' 
\end{enumerate}


\section{Fundamental types}
ExprTk supports three fundamental types which can be used freely in
expressions. The types are as follows:

\begin{description}
\item[Scalar Type]
The scalar type  is a singular  numeric value. The  underlying type is
that used  to specialise  the ExprTk  components (float,  double, long
double, MPFR et al).


\item[Vector Type]
The vector type is a fixed size sequence of contiguous scalar  values.
A  vector  can be  indexed  resulting in  a  scalar value.  Operations
between a vector and scalar will result in a vector with a size  equal
to that  of the  original vector,  whereas operations  between vectors
will result in a  vector of size equal  to that of the  smaller of the
two. In both mentioned cases, the operations will occur element-wise.


\item[String Type]
The string type is a variable length sequence of 8-bit chars.  Strings
can be  assigned and  concatenated to  one another,  they can  also be
manipulated via sub-ranges using the range definition syntax.  Strings
however can not interact with scalar or vector types.

\end{description}
