\chapter{Reference}

\section{Operations}\label{operations}

\begin{description}

\item[add +]\label{op:add} Add multiple numbers together. The input ports allow
multiple wires, which are all summed. If an input port is unwired, it
is equivalent to setting it to zero.

\item[subtract $-$]\label{op:subtract} Subtract two numbers. The input ports
allow multiple wires, which are summed prior to the subtraction being
carried out. If an input port is unwired, it is equivalent to setting
it to zero. Note the small `+' and `-' signs on the input port
indicating which terms are aded or subtracted from the result.

\item[multiply $\times$]\label{op:multiply} Multiple numbers with each other. The input ports
allow multiple wires, which are all multiplied together. If an input
port is unwired, it is equivalent to setting it to one.

\item[divide $\div$]\label{op:divide} Divide a number by another. The input ports allow
multiple wires, which are multiplied together prior to the division
being carried out. If an input port is unwired, it is equivalent to
setting it to one. Note the small `$\times$' and `$\div$' signs indicating which
port refers to the numerator and which the denominator.

\item[log]\label{op:log} Take the logarithm of the $x$ input port, to
base $b$. The base $b$ needs to be specified --- if the natural
logarithm is desired ($b=e$), use the \htmlref{ln operator}{op:ln} instead.

\item[pow $x^y$]\label{op:pow} Raise one number to the power of another. The
ports are labelled $x$ and $y$, referring the the formula $x^y$.

\item[lt $<$]\label{op:lt} Returns 0 or 1, depending
  on whether $x<y$ is true or false.

\item[le $\le$]\label{op:le} Returns 0 or 1, depending
  on whether $x\le y$ is true or false.

\item[eq $=$]\label{op:eq} Returns 0 or 1, depending
  on whether $x=y$ is true or false.

\item[min]\label{op:min} Returns the minimum of $x$ and $y$.

\item[max]\label{op:max} Returns the maximum of $x$ and $y$.

\item[and $\wedge$]\label{op:and_} Logical and of $x$ and $y$, where $x<0.5$
  means false, and $x>0.5$ means true. The output is 1 or 0, depending
  on the result being true or false respectively.

\item[or $\vee$]\label{op:or_} Logical or of $x$ and $y$, where $x<0.5$
  means false, and $x>0.5$ means true. The output is 1 or 0, depending
  on the result being true or false respectively.

\item[not $\neg$]\label{op:not_} The output is 1 or 0, depending
  on whether $x<0.5$ is true or false respectively.

\item[time]\label{op:time} Returns the current value of system time.

\item[copy]\label{op:copy} This just copies its input to its output,
which is redundant on wiring diagrams, but is needed for internal
purposes.

\item[integrate]\label{op:integrate} Creates an integration (or stock)
variable. Editable attributes include the variable's name and its
initial value at $t=0$.

\item[differentiate]\label{op:differentiate} Symbolically differentiates its input.

\item[data]\label{op:data} A data interpolation widget. Currently, the
data must be imported from a file containing two values on each line,
eg:
\begin{tabular}{rr}
0.1 &0.3\\
0.5 &0.7\\
0.9 &1\\
\end{tabular}
If the input is less than the minimum key value (0.1 here), then the
operation outputs the corresponding value (0.3). Similarly if the
input is greater than the maximum (0.9), the corresponding value (1)
is output. If it lies in between two keys (eg 0.2), the the output is
linearly interpolated (0.4).

More formally, a data block is an empirical function, based on a table
of pairs of values ($x_i, y_i, i=1\ldots n, x_{i+1}>x_i$) read in from
a file. The function's output is linearly interpolated from the data,
ie:
\begin{displaymath}
f(x) = \left\{
\begin{array}{cl}
y_1 & x < x_1\\
y_n & x\geq x_n\\
\frac{y_i(x_{i+1}-x)+y_{i+1}(x-x_i)}{x_{i+1}-x_i} & x_i \leq x <
x_{i+1}\\
\end{array}
\right.
\end{displaymath}

\item[sqrt $\surd$]\label{op:sqrt} Square root of the input

\item[exp]\label{op:exp} Exponential of the input

\item[ln]\label{op:ln} Natural logarithm

\item[sin]\label{op:sin} sine function
\item[cos]\label{op:cos} cosine function
\item[tan]\label{op:tan} tangent function
\item[asin]\label{op:asin} Arc sine, inverse of sine
\item[acos]\label{op:acos} Arc cosine, inverse of cosine
\item[atan]\label{op:atan} Arc tangent, inverse of tangent
\item[sinh]\label{op:sinh} hyperbolic sine function $\frac{e^x-e^{-x}}2$
\item[cosh]\label{op:cosh} hyperbolic cosine function $\frac{e^x+e^{-x}}2$
\item[tanh]\label{op:tanh} hyperbolic tangent function $\frac{e^x-e^{-x}}{e^x+e^{-x}}$
\item[abs $|x|$]\label{op:abs} absolute value function
\item[floor $\lfloor x\rfloor$]\label{op:floor} The greatest integer
  less than or equal to $x$.
\item[frac]\label{op:frac} Fractional part of $x$, ie $x-\lfloor x\rfloor$. 

\end{description}

\section{Switch}\label{switch}

A switch block (also known as a case block, or select in the Fortran
world) is a way of selecting from a range of alternatives according
to the value of the input, effectively defining a piecewise function.

\fwhtmladdimg{switch.png}
\begin{center}
{\em An example switch block with 3 cases}
\end{center}

The default switch has two cases, and can be used to implement an
if/then/else construct. However, because the two cases are 0 and 1,
or false and true, a two case switch statement will naturally appear
``upside down'' to how you might think of an if statement. In other
words, it looks like:
{\tt if not }{\em condition} {\tt then}\\
 \ldots
{\tt else}\\
\ldots

You can add or remove cases through the context menu. 

\section{Variables}

Variables represent values in a calculation, and come in a number of
varieties:
\begin{description}
\item[Constants] represent an explicit numerical value, and do not
have a name. Their graphical representation shows the actual value of
the constant.
\item[Parameters] are named constants. All instances of a given name
represent the same value, as with all other named variables, so
changing the value of one parameter, either through its edit menu, or
through a slider, will affect all the others of that name.
\item[Flow variables] have an input port that defines how the value is
to be calculated. Only one flow variable of a given name can have its
input port connected, as they all refer to the same quantity. If no
input ports are connected, then flow variables act just like
parameters.
\item[Integral variables] represent the result of integrating its
input over time  by means of the differential
equation solver. The integrand is represented by the input to an
integral operator that is attached to the integral variable.
\item[Stock variables] are the columns of Godley tables, and represent
the integral over time of the sum of the flow variables making up the
column.
\end{description}

Variables may be converted between types in the variable edit menu,
available from the context menu.

\subsection{Variable names}

Variable names uniquely identify variables. Multiple icons on the
canvas may have the same name --- they all refer to the same
variable. Variable names have scope, which is either global (indicated
by a leading `:', or the numerical id of a group. You may select a
variable name from a drop down list in the ``name'' combo box.

\subsection{Initial conditions}\label{var:init}

Variable initial conditions can be defined through the ``init value''
field of the variable edit menu, or in the case of Godley table stock
variables, through the initial condition row of the Godley table. An
initial value can be a simple number, or it can be a multiple of
another named variable (or parameter). In case of symbolic
definitions, it would be possible to set up a circular reference where
the initial value of variable A is defined in terms of the initial
value of variable B, which in turn depends on the intial value of
A. Such a pathological situation is detected when the system is reset.

\subsection{Sliders}

From the context menu, one can select a slider to be attached to a
variable, which is a GUI ``knob'' allowing one to control a variable's
initial value, or the value of a parameter or constant. Adjusting the
slider of an integral (or stock) variable while the system is running
actually adjusts the present value of the variable.

Slider parameters are specified in the edit menu: max, min and step
size. A relative slider means that the step size is expressed as a
fraction of max-min.

\section{Wires}

Wire represent the flow of values from one operation to the next. To
add a wire to the canvas, click on the output port of an operation or
variable (right hand side of the icon in its initial unrotated
orientation), and then drag it towards an input port (on the left hand
side of an unrotated icon). You can't connect an operator to itself
(that would be a loop, which is not allowed, unless passing through an
integral), nor can an input port have more than one wire
attached, with the exception of $+/-$ and $\times/\div$, where the
multiple wires are summed or multiplied, respectively, and similarly max/min.

Wires can be bent by dragging the blue dots (``handles''). Every time
a handle is dragged out of a straight line with its neighbours, new
handles appear on either side. Handles can be removed by
double-clicking on them.

\section{Groups}\label{group}

Grouping gives the capability to create reusable modules, or subroutines that
can dramatically simplify more complicated systems. Groups may be
created in the following ways:
\begin{itemize}
\item by lassoing a number of items to select them, then selecting
"group" from the canvas context menu, or the edit menu.
\item by pasting the selection
\item by copying another group
\item by inserting a Minsky file as a group
\end{itemize}

Zooming in on a group allows you see and edit its contents. Groups may
be nested heirarchically, which gives an excellent way of zooming in
to see the detail of a model, or zooming out to get an overview of it.

Around the edges of a group are input or output variables, which allow
one to parameterise the group. One can drag a variable and dock it in
the I/O area to create a new input or output for the group.

When creating a group, or dragging a variable or operation into or out
of a group, if a wire ends up crossing the group boundary, a new
temporary variable is added as an I/O variable.

Variable names within groups are locally scoped to that group. That
means that a variable of the same name outside the group refers to a
different entity completely. One can refer to variables outside the
current scope by qualifying the variable name. The simplest
qualification to understand is global scope, which is specified by
prepending a `:' to the variable name. To refer to a variable in
another group, the notation is \verb+<group id>:<name>+, where
\verb+<group id>+ is the integer id of the group in question. An
alternative equivalent notation is \verb+<description>[<group id>]:<name>+, where \verb+<description>+ is the textual name given to
the group. The description string is ignored by Minky on input - only
the group id is important, but this mechanism makes it easier to see
what variable belongs to which group in the drop down lists in the
name field. Please try to keep the group names distinctive in their
first five characters, as Minsky will truncate the descriptive string
to fit in the drop down menu. A group can also be exported to a file from the context menu.
This allows you to build up a library of building blocks.

\section{Plot widget}
\label{Plot}

A plot widget embeds a dynamic plot into the canvas. Around the
outside of the plot are a number of input ports that can be wired.

\htmladdimg{plotWidget.png}

\begin{description}
\item[left hand edge] Up to 4 quantities can be plotted on the graph
  simultaneously, with line colour given by the colour of the input
  port
\item[right hand edge] Another 4 quantities can be added to the
  plot. These are shown on a different scale to the left hand inputs,
  allowing very different magnitudes to be compared on the one plot.
\item[bottom edge] Quantities controlling the x-coordinates of the
  curves. The colours match up with the colour of the pen being
  controlled.

 \fwhtmladdimg{plotLissajous.png}

  If only one bottom port is connected, then that controls all pens
  simultaneously, and if no ports are connected, then the simulation
  time is used to provide the x coordinates
\item[corners] Corner ports control the scale. You can wire up
  variables controlling minimum and maximum of the x, y and right hand
  y axes. If left unwired, the scales are determined automatically
  from the data. This can be used, for example, to implement a sliding
  window graph

 \fwhtmladdimg{plotSlidingWindow.png}
\end{description}

\section{Note Widget}
 \label{Notes} Notes allow arbitrary text to be
placed on the canvas for explanatory purposes. Anything that can be
entered on the keyboard can be placed here, including unicode
characters, but LaTeX formatting is not currently supported. A note
widget, like all canvas items, allow short additional tooltips to be
specified (not in Mun!). 

\section{Godley Tables}\label{godley}

Godley tables describes sets of financial flows from the point of view
of a particular economic agent, such as a bank. The columns of the
table represent accounts (possibly aggregated), which are treated as
integration variables by the system. In ``double entry'' mode,
accounts may be assets, liabilities or equities. Assets may appear as
liabilities in another agents Godley table, and vice versa, with the
sense of the financial flows treated oppositely (a positive flow increasing the
asset of one entity will appear as a negative flow, increasing the
value of a liability). Instead of positive or negative flows, one can
optionally use CR and DR prefixes, as specified in the options panel.

The first row specifies the stock variables, after which follow the
flow rows. Usually, the row marked ``Initial Conditions'' comes next,
but may be placed in any position. These specify the initial
conditions of the stock variables, and may refer to a multiple of
another variable, just like the \htmlref{initial condition
field}{var:init}, or just be a numerical value.

Finally come the flows. The first column is a simple textual label
(the word ``Initial Conditions'', regardless of capitalisation, is a
reserved word for setting stock variable initial conditions)
identifying the flow. The flows themselves are written as a numerical
multipler times a flow variable. Unscoped variables are treated as
global at present, however, in the future, Godley tables will be
allowed to be part of groups, which will then define the context of
the unqualified variable names.

The final column displays the row sum. A correctly functioning Godley
table should have each row sum to zero --- this ensures everything is
accounted for, with no hidden sinks or sources.

\section{Context Menu}

All canvas items have a context menu, which allow a variety of
operations to be applied to the canvas item. Common context menu items
are explained here:
\begin{description}
\item[Help] bring up context specific help for the item
\item[Description] Attach an annotation to the item. This is only
visible by selecting the description item from the context menu,
although whatever is set as the ``Short Description'' will also appear
as a tooltip whenever the mouse hovers over the item.
\item[Port values] When running a simulation, you can drill down into
the actual values at the input and output ports of the variable or
operation, which is a useful aid for debugging models.
\item[Edit] set or query various attributes of an item. This function
can also be accessed by double clicking on the item. (Plot widgets
behave slightly differently).
\item[Copy] Creates a copy of an item, retaining the same attributes
of the original. This is very useful for creating copies of the same
variable to reduce the amount of overlapping wiring (aka "rats nest")
in a model.
\item[Flip] actually rotates an object through $180^\circ$. You can
specify aribtrary rotations of objects through the edit menu.
\item[Raise/Lower] Raise and lower the canvas items relative to each
other. You may need to do this if a large item such as a Godley table
or plot is obscuring a wire, making it hard to access the wire's
context menu or handles,
\item[Browse object] gives a low level drilldown of the internal C++
object this canvas item represents. It is perhaps more of interest to
developers. 
\item[Delete] delete the object.
\end{description}

Item specific context menu items:
\begin{description}
\item[variables, parameters and constants]
\begin{description}
\item[Slider] add a slider control to a variable. This is most
effective for controlling parameters and constants, but can also be
used to control inputless variables.
\item[Add integral] attach an integration operation, and convert the
variable into an integral type
\end{description}
\item[integrals]
\begin{description}
\item[Copy Var] copy just the integration variable, not the
integration operation
\item[Toggle Var Binding] Normally, integrals are tightly bound to their
variables. By toggling the binding, the integral icon can then be
moved independently of the variable it is bound to. 
\end{description}
\item[Godley tables]
\begin{description}
\item[Open Godley Table] opens a spreadsheet to allow financial flows
defining the Godley table to entered or modified.
\item[Edit/Copy var] allows individual stock and flow variables to be
copied or edited.
\end{description}
\item[Plot Widgets]
\begin{description}
\item[Expand]
By double-clicking, or selecting ``Expand'' from the context menu, a
popup window is created of the plot, which can be used examine the
plotting in more detail.

\item[Resize] Allows you to resize the plot icon on the canvas
\item[Options] Customize the plot by adding a title, axes labels and
  control the number of axis ticks and grid lines on the detailed
  plot. You can also add a legend, which is populated from the names
  of variables attached to the plot.
\item[Delete] remove the plot
\end{description}

\end{description}

\section{Canvas background}

The canvas is not simply an inert place for the canvas items to
exist. There is also a background context menu, giving access to the
edit menu functionality, and also keyboard entry.

The following keystrokes insert an operation

\begin{tabular}{rl}
\verb-+- & add\\
\verb+-+ & subtract \\
\verb+*+ & multiply\\
\verb++/ & divide\\
\verb+^+ & pow\\
\verb+&+ & integral\\
\verb+=+ & Godley table\\
\verb+@+ & plot\\
\verb+%+ or \verb+#+ & start a text comment, finish with return\\
\end{tabular}

Typing any other character, then return will insert an operation (if
the name matches), or otherwise a variable with that name.
