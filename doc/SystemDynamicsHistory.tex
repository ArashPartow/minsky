\chapter{Brief History of system dynamics}

ystem dynamics programs express equations using a visual
metaphor. Before this metaphor was developed, anyone who wanted to
simulate a system or procedure had to write a computer program
(normally in
\htmladdnormallinkfoot{FORTRAN}{https://en.wikipedia.org/wiki/FORTRAN}). The
program might well replicate the system accurately, but only the
author of the program would know that—--everyone else had to take it on
faith, because the program code would be indecipherable to everyone
but the author.



For example, below is the Wikipedia’s example of a simple FORTRAN II
program to calculate the area of a triangle. Enjoy the ``My Eyes Glaze
Over'' effect of trying to understand what it does at your leisure.

\begin{verbatim}
LISTING
     READ INPUT TAPE 5, 501, IA, IB, IC
 501 FORMAT (3I5)
     IF (IA) 777, 777, 701
 701 IF (IB) 777, 777, 702
 702 IF (IC) 777, 777, 703
 703 IF (IA+IB-IC) 777,777,704
 704 IF (IA+IC-IB) 777,777,705
 705 IF (IB+IC-IA) 777,777,799
 777 STOP 1

 799 S = FLOATF (IA + IB + IC) / 2.0
     AREA = SQRT( S * (S - FLOATF(IA)) * (S - FLOATF(IB)) *
    +     (S - FLOATF(IC)))
     WRITE OUTPUT TAPE 6, 601, IA, IB, IC, AREA
 601 FORMAT (4H A= ,I5,5H  B= ,I5,5H  C= ,I5,8H  AREA= ,F10.2,
    +        13H SQUARE UNITS)
     STOP
     END
\end{verbatim}

A more complex problem would involve hundreds of pages of computer
code that no-one could understand (bar, one hoped, the person who
wrote it). \htmladdnormallinkfoot{Jay
Forrester}{https://en.wikipedia.org/wiki/Jay\_Forrester}, an engineer
who had designed
\htmladdnormallinkfoot{Whirlwind}{https://en.wikipedia.org/wiki/Whirlwind\_(computer)}(the
first computer with a video display for output), was confronted with
such a problem: working out for GE why a Kentucky appliance factory
had a regular three-year cycle in its employment. He developed a
diagrammatic representation of the manufacturing and hiring processes
at the factory, and showed that its internal hiring practices
generated the cycle.


System dynamics was thus born as a visual way of understanding a
complex system. With his background in developing computers that
produced visual output on a
\htmladdnormallinkfoot{CRT}{https://en.wikipedia.org/wiki/Cathode\_ray\_tube}
(``cathode ray tube''; before Whirlwind---and for some time after
it---most computers printed their results on line printers), it was a
logical step for Forrester to move from hand-drawn diagrams to visual
representations of a complex model on a computer screen. 

The first such program had a hoot of a name: SIMPLE, which stood for
``Simulation of Industrial Management Problems with Lots of Equations''
(it therefore qualifies for the annual COCOA award, given by the
\htmladdnormallinkfoot{Committee to Outlaw Contrived and Outrageous
Acronyms}{https://www.acronymfinder.com/Committee-to-Outlaw-Contrived-and-Outrageous-Acronyms-(COCOA).html}). After
SIMPLE came
\htmladdnormallinkfoot{DYNAMO}{https://en.wikipedia.org/wiki/DYNAMO\_(programming\_language)}
(based on the slightly less outrageous acronym DYNAmic MOdels), and
ultimately the proliferation of modern commercial and Open Source
system dynamics programs. 

Now there are many programs---both commercial and Open Source---which
have built on the initial ideas in SIMPLE and DYNAMO, and they are the
workhorse tools of engineers all over the world. Here, for example, is
a simple model of a 4 cylinder car (one of the example files supplied
with Vissim). Real-world projects are substantially more complicated,
but this gives an idea of the complexity of tasks that can be
modelled, and the capability for breaking a complex system into
sub-units for design purposes. 

\fwhtmladdimg{NewItem2.png}

Management, marketing and social science disciplines use a related
class of programs, like Stella. Here is a marketing model of customers
switching from one product to another: 

\htmladdimg{NewItem3.png}


